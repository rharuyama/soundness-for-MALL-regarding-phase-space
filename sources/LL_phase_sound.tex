\documentclass[dvipdfmx,cjk]{beamer} 

\AtBeginDvi{\special{pdf:tounicode 90ms-RKSJ-UCS2}}

\usetheme{CambridgeUS}

\usepackage{bussproofs} %% For proof tree
\usepackage{ cmll } %% For connective ``par''
\usepackage[all,cmtip]{xy} %% For Hasse diagram of LL
\usepackage{amsmath}

\setbeamertemplate{theorems}[numbered]
\newtheorem{thm}{Theorem}[section]
\newtheorem{proposition}[thm]{Proposition}
\theoremstyle{example}
\newtheorem{exam}[thm]{Example}
\newtheorem{remark}[thm]{Remark}
\newtheorem{question}[thm]{Question}
\newtheorem{prob}[thm]{Problem}

%%%%%%%%%%%%%%%%%%%%%%%%%%%%%%%%%%%%%%%%%%%%%%%%%%%%%%%%%%%%%%%%%%%%%%%%%%%%%%%%%%%%%%%%
%%%%%%%%%%%%%%%%%%%%%%%%%%%%%%%%%%%%%%%%%%%%%%%%%%%%%%%%%%%%%%%%%%%%%%%%%%%%%%%%%%%%%%%%

\begin{document}
\title[LL]{Soundness for Linear Logic regarding Phase Smantics} 
\author[Haruyama]{Ryo Haruyama}
\institute[Nagoya Univ.]{Nagoya Univ.}
\date{\today}

\begin{frame}
  \titlepage
\end{frame}

\begin{frame}
  One kind of soundness for Linear Logic as to Phase Semantics is dealt with.
  \tableofcontents
\end{frame}

%%%%%%%%%%%%%%%%%%%%%%%%%%%%%%%%%%%%%%%%%%%%%%%%%%%%%%%%%%%%%%%%%%%%%%%%%%%%%%%%%%%%%%%%
\section{What is Linear Logic?}
%%%%%%%%%%%%%%%%%%%%%%%%%%%%%%%%%%%%%%%%%%%%%%%%%%%%%%%%%%%%%%%%%%%%%%%%%%%%%%%%%%%%%%%%
                          
\begin{frame}
  \frametitle{What is Linear Logic?}
  Features:

  \begin{itemize}
    \item Decomposition of Classical Logic 
    \item Realizing constructivity keeping Duality (symmetry) intact 
    \item Resource sensitive (i.e. each hypothesize can be used exactly at once) 
    \item Suitable for expressing parallel computing
  \end{itemize}
  
\end{frame}

\begin{frame}
  %%%%%%%%%%%%%%%%%%%%%%%%%%%%%%%%%%%%%%%%%%%%%%%%%%%%
  \frametitle{Decomposition of Classical Logic} 
  %%%%%%%%%%%%%%%%%%%%%%%%%%%%%%%%%%%%%%%%%%%%%%%%%%%%

    Familiar connectives ``$\wedge$'' and ``$\vee$'' break into weaker four connectives:

    \begin{itemize}
      \item ``$\wedge$'' into ``$\otimes$'' and ``$\with$'' 
      \item ``$\vee$''   into ``$\parr$''   and ``$\oplus$'' 
    \end{itemize}

    Another viewpoint:

    \begin{itemize}
      \item \textit{Multiplicative} are ``$\otimes$'' and ``$\parr$'' 
      \item \textit{Additive} are ``$\with$''   and ``$\oplus$'' 
    \end{itemize}

    System containing only Multiplicative and Additive is called \textit{MALL}
    
\end{frame}

\begin{frame}
  %%%%%%%%%%%%%%%%%%%%%%%%%%%%%%%%%%%%%%%%%
  \frametitle{Resource sensitiveness} 
  %%%%%%%%%%%%%%%%%%%%%%%%%%%%%%%%%%%%%%%%%

  ``$\multimap$'' is linear version of ``$\Rightarrow$'' \\~\\

  Usual logic :
  \begin{prooftree}
      \AxiomC{$X$}
      \AxiomC{$X \Rightarrow Y$}
      \BinaryInfC{$Y$ (but $X$ still holds.)}
  \end{prooftree} 

  Linear Logic :
  \begin{prooftree}
      \AxiomC{$X$}
      \AxiomC{$X \multimap Y$}
      \BinaryInfC{$Y$ (and then $X$ disappears!)}
  \end{prooftree}

\end{frame}

\begin{frame}
  %%%%%%%%%%%%%%%%%%%%%%%%%%%%%%%%%%%%%%%%%%%%%%%%%%%%%%%%%%%%%%
  \frametitle{$\otimes$ : Tensor, suitable for expressing parallel computing} 
  %%%%%%%%%%%%%%%%%%%%%%%%%%%%%%%%%%%%%%%%%%%%%%%%%%%%%%%%%%%%%%

  \begin{prooftree}

    \AxiomC{$f : A \multimap B$}
    \AxiomC{$g : C \multimap D$}
    \BinaryInfC{$f \otimes g : A \otimes B \multimap C \otimes D$}
    
  \end{prooftree} 

  Meaning : Programs which do NOT share the same resouce can be executed
  at the same time.

\end{frame}

\begin{frame}
  %%%%%%%%%%%%%%%%%%%%%%%%%%%%%%%%%%%%%%%%%%%%%%%%%%%%%%%%%%%%%%
  \frametitle{$\with$ : Cartesian product} 
  %%%%%%%%%%%%%%%%%%%%%%%%%%%%%%%%%%%%%%%%%%%%%%%%%%%%%%%%%%%%%%
  
  \begin{prooftree}

    \AxiomC{$f : X \multimap Y$}
    \AxiomC{$g : X \multimap Z$}
    \BinaryInfC{$f \with g : X \multimap Y \with Z$}
    
  \end{prooftree}

  and

  \begin{prooftree}

    \AxiomC{$Y \with Z \nvdash Y \otimes Z$}
    
  \end{prooftree}

  Meaning : Programs which share the same resouce CANNOT be executed
  at the same time unless copying the resource,
  while we CAN CHOOSE either $Y$ or $Z$. \\
  In fact, copying and deleting of resources are explicit.

\end{frame}

\begin{frame}
  %%%%%%%%%%%%%%%%%%%%%%%%%%%%%%%%%%%%%%%%%%%%%%%%%%%%%%%%%%%%%%
  \frametitle{$\oplus$ : Additive Sum, dual of ``$\with$''} 
  %%%%%%%%%%%%%%%%%%%%%%%%%%%%%%%%%%%%%%%%%%%%%%%%%%%%%%%%%%%%%%
  
  \begin{prooftree}

    \AxiomC{$Y \oplus Z \nvdash Y$}
    
  \end{prooftree}

  nor

  \begin{prooftree}

    \AxiomC{$Y \oplus Z \nvdash Z$}
    
  \end{prooftree} 

  Meaning : Either $Y$ or $Z$ holds, but we CANNOT choose neither $Y$ or $Z$.

\end{frame}

\begin{frame}
  %%%%%%%%%%%%%%%%%%%%%%%%%%%%%%%%%%%%%%%%%%%%%%%%%%%%%%
  \frametitle{Off topic: Variants of Linear Logic}
  %%%%%%%%%%%%%%%%%%%%%%%%%%%%%%%%%%%%%%%%%%%%%%%%%%%%%%

  \begin{itemize}
      \item Multiplicative are ``$\otimes$'' and ``$\parr$'' 
      \item Additive are ``$\with$''   and ``$\oplus$''
      \item \textit{Exponential} are ``$!$'' and ``$?$''
  \end{itemize}

  \[
      \xymatrix{
        & MAELL \ar@{-}[dr] \ar@{-}[dl] \\
        MALL \ar@{-}[dr] & & MELL \ar@{-}[dl] \\
        & MLL
      }
   \]      
  
\end{frame}

%%%%%%%%%%%%%%%%%%%%%%%%%%%%%%%%%%%%%%%%%%%%%%%%%%%%%%%%%%%%%%%%%%%%%%%%%%%%%%%%%%%%%%%%
\section{What is Phase Semantics?}
%%%%%%%%%%%%%%%%%%%%%%%%%%%%%%%%%%%%%%%%%%%%%%%%%%%%%%%%%%%%%%%%%%%%%%%%%%%%%%%%%%%%%%%%
                          
\begin{frame}
  %%%%%%%%%%%%%%%%%%%%%%%%%%%%%%%%%%%%%%%                          
  \frametitle{What is Phase Semantics?}
  %%%%%%%%%%%%%%%%%%%%%%%%%%%%%%%%%%%%%%%
  First off, what is \textit{semantics}?

  \begin{itemize}
    \item To provide rigorous defnitions that abstract away from implementation
      details
    \item To provide mathematical tools for proving properties of programs
  \end{itemize}
   (Amadio, Curien)\\~\\

  Phase Semantics is a kind of semantics, which is based on the idea of \textit{Tarskian style}: \\

  \begin{itemize}
    \item ``A'' means A which is truth value (true or false)
    \item ``A $\wedge$ B'' means ``A'' and ``B''
  \end{itemize}

  and so on.\\

  This seems ovious, however, there is another semantics which is not the case: \textit{Coherent Semantics}, which is BHK style inconsistent semantics. \\

  \textit{Phase Space} is Phase Semantics for MALL.
  
\end{frame}

\begin{frame}
  %%%%%%%%%%%%%%%%%%%%%%%%%%%%%%%%%
  \frametitle{Other semantics}
  %%%%%%%%%%%%%%%%%%%%%%%%%%%%%%%%%
\end{frame}

%%%%%%%%%%%%%%%%%%%%%%%%%%%%%%%%%%%%%%%%%%%%%%%%%%%%%%%%%%%%%%%%%%%%%%%%%%%%%%%%%%%%%%%%
\section{What is soundness?}
%%%%%%%%%%%%%%%%%%%%%%%%%%%%%%%%%%%%%%%%%%%%%%%%%%%%%%%%%%%%%%%%%%%%%%%%%%%%%%%%%%%%%%%%
                          
\begin{frame}
  \frametitle{What is \textit{soundness}?} 

  Formulae derived using specific rules are semantically valid, which is minimum requirement for semantics in general (Systems which yeilds lie are compeletely useless).
  
\end{frame}

%%%%%%%%%%%%%%%%%%%%%%%%%%%%%%%%%%%%%%%%%%%%%%%%%%%%%%%%%%%%%%%%%%%%%%%%%%%%%%%%%%%%%%%%
\section{Proof of soundness}
%%%%%%%%%%%%%%%%%%%%%%%%%%%%%%%%%%%%%%%%%%%%%%%%%%%%%%%%%%%%%%%%%%%%%%%%%%%%%%%%%%%%%%%%

\begin{frame}
  %%%%%%%%%%%%%%%%%%%%%%%%%%%%%%%%%
  \frametitle{Syntax of MALL}
  %%%%%%%%%%%%%%%%%%%%%%%%%%%%%%%%
  \begin{definition}{Formula of MALL.}
    \begin{align*}
      A ::&= p              \; | \;
             p^{\bot}        \\
      &| \;
             A \otimes A    \; | \;
             A \oplus  A    \\
      &| \;
             A \with A      \; | \;
             A \parr A      \\
      &| \;
             \textbf{1}     \; | \;
             \textbf{0}     \; | \;
             \top           \; | \;
             \bot
    \end{align*}
  \end{definition}
\end{frame}

\begin{frame}
  %%%%%%%%%%%%%%%%%%%%%%%%%%%%%%%%%%%%%
  \frametitle{Inference rules of MALL}
  %%%%%%%%%%%%%%%%%%%%%%%%%%%%%%%%%%%%%
  \begin{definition}{Inference rules of MALL}
    \begin{center}
      
      \RightLabel{(\textit{identity})}
      \AxiomC{}
      \UnaryInfC{$ \vdash A, A^\bot $}
      \DisplayProof
      \hskip 1.5em
      \RightLabel{(\textit{cut})}
      \AxiomC{$\vdash \Gamma, A$}
      \AxiomC{$\vdash A^\bot, \Delta$}
      \BinaryInfC{$\vdash \Gamma, \Delta$}
      \DisplayProof
      
    \end{center}
  \end{definition}
\end{frame}

\begin{frame}
  %%%%%%%%%%%%%%%%%%%%%%%%%%%%%%%%%%%%
  \frametitle{Proof of soundness}
  %%%%%%%%%%%%%%%%%%%%%%%%%%%%%%%%%%%%
  \begin{theorem}
    Formulae which are provable in MALL are all valid in Phase Space \\
    i. e. \[
        \vdash \underbar{X} \Rightarrow 1 \in \underbar{X}
    \]
        
  \end{theorem}

  \begin{proof}
    By straitforward induction on inference rules.
  \end{proof}
  
\end{frame}

%%%%%%%%%%%%%%%%%%%%%%%%%%%%%%%%%%%%%%%%%%%%%%%%%%%%%%%%%%%%%%%%%%%%%%%%%%%%%%%%%%%%%%%%
\section{Future work}
%%%%%%%%%%%%%%%%%%%%%%%%%%%%%%%%%%%%%%%%%%%%%%%%%%%%%%%%%%%%%%%%%%%%%%%%%%%%%%%%%%%%%%%%

\begin{frame}
  \frametitle{Future work}

\end{frame}
                          
%%%%%%%%%%%%%%%%%%%%%%%%%%%%%%%%%%%%%%%%%%%%%%%%%%%%%%%%%%%%%%%%%%%%%%%%%%%%%%%%%%%%%%%%
%%%%%%%%%%%%%%%%%%%%%%%%%%%%%%%%%%%%%%%%%%%%%%%%%%%%%%%%%%%%%%%%%%%%%%%%%%%%%%%%%%%%%%%%

\end{document}
