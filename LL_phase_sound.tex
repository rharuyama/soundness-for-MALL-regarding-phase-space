\documentclass[dvipdfmx,cjk]{beamer} 
%\documentclass[dvipdfm,cjk]{beamer} %% オプションは環境や利用するプログラムに
%\documentclass[dvips,cjk]{beamer}   %% よって変える

\AtBeginDvi{\special{pdf:tounicode 90ms-RKSJ-UCS2}} %% しおりが文字化けしないように
%\AtBeginDvi{\special{pdf:tounicode EUC-UCS2}}

%\setbeamertemplate{navigation symbols}{} %% 右下のアイコンを消す

\usetheme{CambridgeUS}         %% theme の選択
%\usetheme{Boadilla}           %% Beamer のディレクトリの中の
%\usetheme{Madrid}             %% beamerthemeCambridgeUS.sty を指定
%\usetheme{Antibes}            %% 色々と試してみるといいだろう
%\usetheme{Montpellier}        %% サンプルが beamer\doc に色々とある。
%\usetheme{Berkeley}
%\usetheme{Goettingen}
%\usetheme{Singapore}
%\usetheme{Szeged}

\usepackage{bussproofs} %% For proof tree
\usepackage{ cmll } %% for connective ``par''

%\usecolortheme{rose}          %% colortheme を選ぶと色使いが変わる
%\usecolortheme{albatross}

%\useoutertheme{shadow}                 %% 箱に影をつける
%\usefonttheme{professionalfonts}       %% 数式の文字を通常の LaTeX と同じにする

%\setbeamercovered{transparent}         %% 消えている文字をうっすらと表示する

\setbeamertemplate{theorems}[numbered]  %% 定理に番号をつける
\newtheorem{thm}{Theorem}[section]
\newtheorem{proposition}[thm]{Proposition}
\theoremstyle{example}
\newtheorem{exam}[thm]{Example}
\newtheorem{remark}[thm]{Remark}
\newtheorem{question}[thm]{Question}
\newtheorem{prob}[thm]{Problem}

%%%%%%%%%%%%%%%%%%%%%%%%%%%%%%%%%%%%%%%%%%%%%%%%%%%%%%%%%%%%%%%%%%%%%%%%%%%%%%%%%%%%%%%%
%%%%%%%%%%%%%%%%%%%%%%%%%%%%%%%%%%%%%%%%%%%%%%%%%%%%%%%%%%%%%%%%%%%%%%%%%%%%%%%%%%%%%%%%


\begin{document}
\title[LL]{Soundness of Phase Semantics on Linear Logic} 
\author[Haruyama]{Ryo Haruyama}
\institute[Nagoya Univ.]{Nagoya Univ.}
\date{April 3, 2020}

\begin{frame}
  \titlepage
\end{frame}

\begin{frame}
  \tableofcontents
\end{frame}

%%%%%%%%%%%%%%%%%%%%%%%%%%%%%%%%%%%%%%%%%%%%%%%%%%%%%%%%%%%%%%%%%%%%%%%%%%%%%%%%%%%%%%%%
                          \section{What is Linear Logic?}
%%%%%%%%%%%%%%%%%%%%%%%%%%%%%%%%%%%%%%%%%%%%%%%%%%%%%%%%%%%%%%%%%%%%%%%%%%%%%%%%%%%%%%%%
                          
\begin{frame}
  \frametitle{What is Linear Logic?}
  Features:

  \begin{itemize}
    \item Decomposition of Classical Logic 
    \item Realizing constructivity keeping Duality (symmetry) intact 
    \item Resource sensitive (i.e. each hypothesize can be used exactly at once) 
    \item Suitable for expressing parallel computing
  \end{itemize}
  
\end{frame}

\begin{frame}
  %%%%%%%%%%%%%%%%%%%%%%%%%%%%%%%%%%%%%%%%%%%%%%%%%%%%
  \frametitle{Decomposition of Classical Logic} 
  %%%%%%%%%%%%%%%%%%%%%%%%%%%%%%%%%%%%%%%%%%%%%%%%%%%%

    Familiar connectives ``$\wedge$'' and ``$\vee$'' break into waeker four connectives:

    \begin{itemize}
      \item ``$\wedge$'' into ``$\otimes$'' and ``$\with$'' 
      \item ``$\vee$''   into ``$\parr$''   and ``$\oplus$'' 
    \end{itemize}
    
\end{frame}

\begin{frame}
  %%%%%%%%%%%%%%%%%%%%%%%%%%%%%%%%%%%%%%
  \frametitle{Resource sensitive} 
  %%%%%%%%%%%%%%%%%%%%%%%%%%%%%%%%%%%%%%

  Usual logic :
  \begin{prooftree}
      \AxiomC{$X$}
      \AxiomC{$X \Rightarrow Y$}
      \BinaryInfC{$Y$ (but $X$ still holds.)}
  \end{prooftree} 

  Linear Logic :
  \begin{prooftree}
      \AxiomC{$X$}
      \AxiomC{$X \multimap Y$}
      \BinaryInfC{$Y$ (and then $X$ disappears!)}
  \end{prooftree}

\end{frame}

\begin{frame}
  %%%%%%%%%%%%%%%%%%%%%%%%%%%%%%%%%%%%%%%%%%%%%%%%%%%%%%%%%%%%%%
  \frametitle{Suitable for expressing parallel computing} 
  %%%%%%%%%%%%%%%%%%%%%%%%%%%%%%%%%%%%%%%%%%%%%%%%%%%%%%%%%%%%%%

  \begin{prooftree}

    \AxiomC{$f : A \multimap B$}
    \AxiomC{$g : C \multimap D$}
    \BinaryInfC{$f \otimes g : A \otimes B \multimap C \otimes D$}
    
  \end{prooftree} 

  Meaning : Programs which do NOT share the same resouce can be executed
  at the same time.

\end{frame}

%%%%%%%%%%%%%%%%%%%%%%%%%%%%%%%%%%%%%%%%%%%%%%%%%%%%%%%%%%%%%%%%%%%%%%%%%%%%%%%%%%%%%%%%
                          \section{What is Phase Semantics?}
%%%%%%%%%%%%%%%%%%%%%%%%%%%%%%%%%%%%%%%%%%%%%%%%%%%%%%%%%%%%%%%%%%%%%%%%%%%%%%%%%%%%%%%%
                          
\begin{frame} 
  \frametitle{What is Phase Semantics?}
  
\end{frame}

%%%%%%%%%%%%%%%%%%%%%%%%%%%%%%%%%%%%%%%%%%%%%%%%%%%%%%%%%%%%%%%%%%%%%%%%%%%%%%%%%%%%%%%%
                          \section{What is Soundness?}
%%%%%%%%%%%%%%%%%%%%%%%%%%%%%%%%%%%%%%%%%%%%%%%%%%%%%%%%%%%%%%%%%%%%%%%%%%%%%%%%%%%%%%%%
                          
\begin{frame}
  \frametitle{What is Soundness?}

  
\end{frame}

%%%%%%%%%%%%%%%%%%%%%%%%%%%%%%%%%%%%%%%%%%%%%%%%%%%%%%%%%%%%%%%%%%%%%%%%%%%%%%%%%%%%%%%%
%%%%%%%%%%%%%%%%%%%%%%%%%%%%%%%%%%%%%%%%%%%%%%%%%%%%%%%%%%%%%%%%%%%%%%%%%%%%%%%%%%%%%%%%

\end{document}
